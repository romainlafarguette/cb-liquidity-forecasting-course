% Time-stamp: <2022-11-04 05:11:40 ubuntu>
% Romain Lafarguette 2020, https://romainlafarguette.github.io/

%% ---------------------------------------------------------------------------
%% Preamble: Packages and Setup
%% ---------------------------------------------------------------------------
% Class 
\documentclass{beamer}

% Theme
\usetheme{Boadilla}
\usecolortheme{dolphin}
%\setbeamertemplate{headline}{} % Remove the top navigation bar

% Font and encoding
\usepackage[utf8]{inputenc} % Input font
\usepackage[T1]{fontenc} % Output font
\usepackage{lmodern} % Standard LateX font
\usefonttheme{serif} % Standard LateX font

% Maths 
\usepackage{amsfonts, amsmath, mathabx, bm, bbm} % Maths Fonts

% Graphics
\usepackage{graphicx} % Insert graphics
\usepackage{subfig} % Multiple figures in one graphic
\graphicspath{{/../static/img}{/../static/diagrams}{/../static/course_1_img}}

% Layout
\usepackage{changepage}

% Colors
\usepackage{xcolor}
\definecolor{imfblue}{RGB}{0,76,151} % Official IMF color
\setbeamercolor{title}{fg=imfblue}
\setbeamercolor{frametitle}{fg=imfblue}
\setbeamercolor{structure}{fg=imfblue}
\setbeamercolor{page number in head/foot}{fg=imfblue}
\setbeamerfont{page number in head/foot}{size=\footnotesize}

% Tables
\usepackage{booktabs,rotating,multirow} % Tabular rules and other macros
\usepackage{pdflscape,afterpage} % Landscape mode and afterpage
\usepackage{threeparttable} % Split long tables
\usepackage[font=scriptsize,labelfont=scriptsize,labelfont={color=imfblue}]{caption}

% Import files
\usepackage{import}

% Appendix slides
\usepackage{appendixnumberbeamer} % Manage page numbers for appendix slides

% References
\usepackage{hyperref}

% A few macros: environments
\newenvironment{wideitemize}{\itemize\addtolength{\itemsep}{10pt}}{\enditemize}
\newenvironment{wideenumerate}{\enumerate\addtolength{\itemsep}{10pt}}{\endenumerate}

\newenvironment{extrawideitemize}{\itemize\addtolength{\itemsep}{30pt}}{\enditemize}
\newenvironment{extrawideenumerate}{\enumerate\addtolength{\itemsep}{30pt}}{\endenumerate}

% Define the footer with higher/lower adjustment
\defbeamertemplate{footline}{higher page number}
{
  \hfill
  \usebeamercolor[fg]{page number in head/foot}
  \usebeamerfont{page number in head/foot}  
  \thepage/\inserttotalframenumber\kern1em\vskip2pt %Change xxpt to
                                %lower/higher the footnote
}
\setbeamertemplate{footline}[higher page number]

% Remove navigation symbols and other superfluous elements
\setbeamertemplate{navigation symbols}{}
\setbeameroption{hide notes}
\setbeamertemplate{note page}[plain]
\beamertemplatenavigationsymbolsempty
\hypersetup{pdfpagemode=UseNone} % don't show bookmarks on initial view

% Institute font
\setbeamerfont{institute}{size=\footnotesize}
\DeclareMathSizes{10}{9}{7}{5}  

%% ---------------------------------------------------------------------------
%% Title info
%% ---------------------------------------------------------------------------
\title[Concepts]{Core Concepts in Financial Econometrics}
\author[R. Lafarguette]{Romain Lafarguette, Ph.D.}
\institute[IMF]{ADIA Quant \& IMF External Consultant}

\date[STI, 08 Nov 2022]{Singapore Training Institute, 08 November 2022}

\titlegraphic{
    \begin{figure}
    \centering
    \subfloat{{\includegraphics[width=2cm]{../static/img/imf_logo}}}%
    \end{figure}
}

%% ---------------------------------------------------------------------------
%% Title slide
%% ---------------------------------------------------------------------------
\begin{document}

\begingroup
\renewcommand{\insertframenumber}{}
\begin{frame}
  %\addtocounter{framenumber}{-1}
\maketitle
\end{frame}
\endgroup


% \begin{frame}
%   \frametitle{Outline}

%   This short crash-course is divided in three parts covering:\\
%   \smallskip
  
%   \begin{enumerate}
%   \item \textbf{Data concepts}: population, sample, data types, data generating process, etc.
%   \item \textbf{Statistical Inference}: stochastic process, estimator, distributions, convergence, etc.
%   \item \textbf{Statistical Properties of Financial Time Series}: stationarity, autocorrelation, heavy tails, etc.
%   \end{enumerate}


% \smallskip
% \emph{NB: this slide-deck follows the excellent course of Christophe Hurlin \url{https://sites.google.com/view/christophe-hurlin/teaching-resources}}  
% \end{frame}




\section{ETS Model}


\begin{frame}{Simple Exponential Smoothing}
Simple exponential smoothing is a weighted average of past observations, with weights decreasing exponentially

\begin{equation*}
  y_{t+1} = \alpha y_t + \alpha(1-\alpha) y_{t-1} + \alpha(1-\alpha)^2 y_{t-2} + \dots 
\end{equation*}

\begin{itemize}
\item More advanced specifications allow for trends, seasonality, etc.
\end{itemize}


\end{frame}


\begin{frame}
  \frametitle{From Simple Exponential Smoothing to ETS}
  
  \begin{itemize}
  \item ETS (Error, Trend, Season) model was developed in the 1950s as algorithms to produce point forecasts
  \item ETS combines a "level" ($l_{t-1}$), a "trend" (($b_{t-1}$)) and a "seasonal" ($s_{t-m}$) components to describe a time series
  \item The rate of change of the components are controlled by "smoothing" parameters: $\alpha$ for the level, $\beta$ for the trend, $\gamma$ for the seasonal
  \item The researcher has to:
    \begin{enumerate}
    \item To choose the best values for the smoothing parameters 
    \item The initial state of the parameters
    \end{enumerate}
  \item Equivalent ETS state-space models have been developped in the 1990s and the 2000s
  \end{itemize}

% Under the simplest, additive specification:
%   \begin{equation*}
%     x_t = l_{t-1} + b_{t-1} + s_{t-m} + \epsilon_t
%   \end{equation*}
  
\end{frame}


\begin{frame}
  \frametitle{Combining Level, Trend and Seasonal Components}

  \begin{itemize}
  \item \textbf{Additively:} $y_t = l_{t-1} + b_{t-1} + s_{t-m} + \epsilon_t$
  \item \textbf{Multiplicatively:} $y_t = l_{t-1} \times b_{t-1} \times s_{t-m} \times (1 + \epsilon_t)$
  \item \textbf{Mixed:} $y_t = (l_{t-1} + b_{t-1}) \times s_{t-m} + \epsilon_t $  
  \end{itemize}
  
\end{frame}




\begin{frame}
  \frametitle{Main Idea: Control the Rate of Change}

  \begin{itemize}
  \item $\alpha$ controls the flexibility of the \textbf{level}
    \begin{itemize}
    \item If $\alpha = 0$, the level never updates (stays at the mean)
    \item If $\alpha = 1$, the level updates completely (naive, start from yesterday)
    \end{itemize}
  \item $\beta$ controls the flexibility of the \textbf{trend}
    \item If $\beta = 0$, the trend is linear
    \item If $\beta = 1$, the trend changes suddenly at each observation
    
    \item $\gamma$ controls the flexibility of the \textbf{seasonality}
      \begin{itemize}
      \item If $\gamma = 0$ the seasonality is fixed (seasonal mean)
      \item If $\gamma = 1$ the seasonality updates completely (seasonal naive)        
      \end{itemize}
  \end{itemize}
  
\end{frame}





\section{ARMA Model}

















\end{document}



%%% Local Variables:
%%% mode: latex
%%% TeX-master: t
%%% End:

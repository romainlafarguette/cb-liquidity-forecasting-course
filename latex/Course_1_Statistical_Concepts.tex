% Time-stamp: <2022-11-02 12:36:07 ubuntu>
% Romain Lafarguette 2020, https://romainlafarguette.github.io/

%% ---------------------------------------------------------------------------
%% Preamble: Packages and Setup
%% ---------------------------------------------------------------------------
% Class 
\documentclass{beamer}

% Font and encoding
\usepackage[utf8]{inputenc} % Input font
\usepackage[T1]{fontenc} % Output font
\usepackage{lmodern} % Standard LateX font
\usefonttheme{serif} % Standard LateX font

% Maths 
\usepackage{amsfonts, amsmath, mathabx, bm, bbm} % Maths Fonts

% Graphics
\usepackage{graphicx} % Insert graphics
\usepackage{subfig} % Multiple figures in one graphic
\graphicspath{{/../static/img}{/../static/diagrams}}

% Layout
\usepackage{changepage}

% Colors
\usepackage{xcolor}
\definecolor{imfblue}{RGB}{0,76,151} % Official IMF color
\setbeamercolor{title}{fg=imfblue}
\setbeamercolor{frametitle}{fg=imfblue}
\setbeamercolor{structure}{fg=imfblue}
\setbeamercolor{page number in head/foot}{fg=imfblue}
\setbeamerfont{page number in head/foot}{size=\footnotesize}

% Tables
\usepackage{booktabs,rotating,multirow} % Tabular rules and other macros
\usepackage{pdflscape,afterpage} % Landscape mode and afterpage
\usepackage{threeparttable} % Split long tables
\usepackage[font=scriptsize,labelfont=scriptsize,labelfont={color=imfblue}]{caption}

% Import files
\usepackage{import}

% Appendix slides
\usepackage{appendixnumberbeamer} % Manage page numbers for appendix slides

% References
\usepackage{hyperref}

% A few macros: environments
\newenvironment{wideitemize}{\itemize\addtolength{\itemsep}{10pt}}{\enditemize}
\newenvironment{wideenumerate}{\enumerate\addtolength{\itemsep}{10pt}}{\endenumerate}

\newenvironment{extrawideitemize}{\itemize\addtolength{\itemsep}{30pt}}{\enditemize}
\newenvironment{extrawideenumerate}{\enumerate\addtolength{\itemsep}{30pt}}{\endenumerate}

% Define the footer with higher/lower adjustment
\defbeamertemplate{footline}{higher page number}
{
  \hfill
  \usebeamercolor[fg]{page number in head/foot}
  \usebeamerfont{page number in head/foot}  
  \thepage/\inserttotalframenumber\kern1em\vskip2pt %Change xxpt to
                                %lower/higher the footnote
}
\setbeamertemplate{footline}[higher page number]

% Remove navigation symbols and other superfluous elements
\setbeamertemplate{navigation symbols}{}
\setbeameroption{hide notes}
\setbeamertemplate{note page}[plain]
\beamertemplatenavigationsymbolsempty
\hypersetup{pdfpagemode=UseNone} % don't show bookmarks on initial view

% Institute font
\setbeamerfont{institute}{size=\footnotesize}
\DeclareMathSizes{10}{9}{7}{5}  

%% ---------------------------------------------------------------------------
%% Title info
%% ---------------------------------------------------------------------------
\title[Concepts]{Introduction to Time Series Econometrics}
\author[R. Lafarguette]{Romain Lafarguette, Ph.D.}
\institute[IMF]{Quant \& IMF External Consultant}

\date[STI, 08 Nov 2022]{Singapore Training Institute, 08 November 2022}

\titlegraphic{
    \begin{figure}
    \centering
    \subfloat{{\includegraphics[width=2cm]{../static/img/imf_logo}}}%
    \end{figure}
}

%% ---------------------------------------------------------------------------
%% Title slide
%% ---------------------------------------------------------------------------
\begin{document}

\begingroup
\renewcommand{\insertframenumber}{}
\begin{frame}
  %\addtocounter{framenumber}{-1}
\maketitle
\end{frame}
\endgroup


\begin{frame}
  \frametitle{Outline}
  \begin{enumerate}
  \item \textbf{Data concepts}: population, sample, data types, data generating process, etc.
  \item \textbf{Estimation strategy}
  \end{enumerate}


\smallskip
\emph{NB: this slide-deck is inspired by the excellent course of \url{https://sites.google.com/view/christophe-hurlin/teaching-resources}{Christophe Hurlin}}  
\end{frame}



%% ---------------------------------------------------------------------------
%% Core
%% ---------------------------------------------------------------------------

\begin{frame}
  \frametitle{Overview}

Financial econometrics (including time-series econometrics) are based on four main elements:\\
\smallskip

  \begin{wideenumerate}
    \item A sample of data
    \item An econometric model, based on a theory or not
    \item An estimation method to estimate the coefficients of the model
    \item Inference/testing approach to validate the estimation
  \end{wideenumerate}
  
\end{frame}



\section{Data Concepts}
\begin{frame}
  \frametitle{Population vs. Sample}

  \begin{block}{Definition: Population}
    A \textbf{population} is defined as including all entities (e.g. banks or firms) or all the time periods of the processus that has to be explained
  \end{block}

\smallskip
  
  \begin{wideitemize}
    \item In most cases, it is impossible to observe the entire statistical population, due to constraints (recording period, cost, etc.)
    \item A researcher would instead observe a \textbf{statistical sample} from the population. He will estimate an econometric model to understand the \textbf{properties on the population as a whole}.
  \end{wideitemize} 
\end{frame}


\begin{frame}
  \frametitle{Data Generating Process}
  
  \begin{block}{Definition: Data Generating Process}
    A \textbf{Data Generating Process (DGP)} is a process in the real world that "generates" the data (or the sample) of interest
  \end{block}

  \begin{exampleblock}{Example: Data Generating Process}
    Let us assume that there is a linear relationship between interest rates in two countries ($R, R^*$), their forward ($F$) and their spot exchange rate ($S$).\\

    \begin{equation*}
      \frac{F}{S} = \frac{1+R}{1+R^*}
    \end{equation*}

    This non-arbitrage relationship (CIP) can be used in the foreign exchange market to determine the forward exchange rate

    \begin{equation*}
      \mathbb{E}[F |S=s, R=r, R^*= r^*] = s*\frac{1+r}{1+r^*}
    \end{equation*}

This relationship is the \textbf{Data Generating Process} for $F$     
  \end{exampleblock}

The equivalent of population for time series econometrics is the DGP.\\
NB: note that I use $R$ to describe the random variable and $r$ to describe its realization
  
\end{frame}


\begin{frame}
  \frametitle{Econometrics Challenge}
  The challenge of econometrics is to draw conclusions about a DGP (or population), after observing only one realization $\{x_1, \dots X_N\}$ of a random sample (the dataset).\\

%TODO: add a diagram
  
\end{frame}


\begin{frame}
  \frametitle{Data Types}

  In econometrics, sets can be mainly distinguished in three types:\\
  \smallskip
  
  \begin{wideenumerate}
    \item Cross-sectional data
    \item Time series data
    \item Panel data
  \end{wideenumerate}
  
\end{frame}

% ADD some diagrams
\begin{frame}
  \frametitle{Cross-Sectional Data}
  Cross-sectional data are the most common type of data encountered in statistics and econometrics.\\ 
  \smallskip
  
  \begin{wideitemize}
    \item Data at the entities level: banks, countries, individuals, households, etc.
    \item \textbf{No time dimension}: only one "wave" or multiple waves of different entities
    \item Order of data does not matter: no time structure
  \end{wideitemize}
\end{frame}


\begin{frame}
  \frametitle{Time Series Data}

  Time series data are very common in financial econometrics and central banking. They entail specific estimation methods to do the \textbf{time-dependence}.\\
  
  \begin{wideitemize}
    \item Data for a single entity (person, bank, country, etc.) collected at multiple time periods. Repeated observations of the same variables (interest rate, GDP, prices, etc.)
    \item Order of data is important!
    \item The observations are typically not independent over time
    \item In this case, the notion of population corresponds to the \textbf{Data Generating Process (DGP)} 
  \end{wideitemize}  
\end{frame}


\begin{frame}
  \frametitle{Panel Data}
  Also called longitudinal data. They contain the most information and allow for more complex estimation and analysis.

  \begin{wideitemize}
    \item Data for multiple entities (individuals, firms, countries, banks, etc.) in which outcomes and characteristics of each entity are observed at multiple points in time
    \item Combine cross-sectional and time-series information
    \item Present several advantages with respect to cross-sectional and time series data, depending on the topic at hands
  \end{wideitemize}

\end{frame}


\begin{frame}
  \frametitle{Econometric Model}

  \begin{block}{Definition: Econometric Model}
    An econometric model specificies the statistical relationship between different economic variables, that are expected to be stable over time
  \end{block}

  \begin{enumerate}
  \item \textbf{Parametric model:} fully characterization of the relationship by a \textbf{set of parameters} $\theta$ and a \textbf{link function} $f$ supposed to be known; the specification can be linear or non linear, and includes some randomness $\epsilon$
    \begin{equation*}
      Y = f(X; \theta) + \epsilon
    \end{equation*}

  \item \textbf{Non-parametric and semi-parametric models}: the link function can not be described using a finite number of parameters. The link function is assumed to be unknown and has to be estimated
    
  \end{enumerate}
  
\end{frame}


\section{Financial Econometrics}

\begin{frame}
  \frametitle{Empirical Strategy}
  The general approach of (financial) econometrics is as follows:\\
  \smallskip

  \begin{wideenumerate}
  \item Specification of the model
  \item Estimation of the parameters
  \item Diagnostic tests
    \begin{itemize}
    \item Significance tests
    \item Specification tests
    \item Backtesting tests
    \item etc.
    \end{itemize}
  \item Interpretation and use of the model
  \end{wideenumerate}
  
\end{frame}

\section{Misc}

\begin{frame}
  \frametitle{Random variable}
    \begin{itemize}
    \item A \textbf{random variable} is a function $f: \Omega \mapsto \mathcal{R}$ that assigns to
      a set of outcome $\Omega$ a \textbf{value}, often a real number.
     \item The probability of an outcome is equal to its \textbf{measure} divided by
        the measure of all possible outcomes
     \begin{itemize}
      \item Example: obtaining an even number by rolling a dice: $\{2, 4, 6\}$
      \item Probability to obtain an even number by rolling a dice: $m(\{2, 4,
        6\})/m(\{1, 2, 3, 4, 5, 6\}) = \frac{1}{2}$ \emph{(here, the measure
          simply "counts" the outcomes with equal weights)}
        \end{itemize}     
    \end{itemize}    

    
  \begin{itemize}
  \item Random variables are the "building block" of statistics:
    \begin{itemize}
    \item Random variables are characterized by their distribution (generating
      function, moments, quantiles, etc.)
    \item The behavior of two or more random variables can be characterized by
      their dependence/independence, matrix of variance-covariance, joint
      distribution, etc.
    \item The main theorem of statistics (law of large numbers, central limit
      theorem, etc.) leverages the properties of random variables
    \end{itemize}
  \end{itemize}
\end{frame}



\begin{frame}
  \frametitle{Stochastic Process}

  \begin{itemize}
  \item A stochastic process is a sequence of random variables indexed by time
    ($t$):
    \begin{equation*}
      {\dots, Y_1, Y_2, \dots, Y_t, Y_{t+1}, \dots} = 
    \end{equation*}
    
  \end{itemize}
  

  
\end{frame}


\begin{frame}
  \frametitle{Stationarity}
  
\end{frame}


\begin{frame}
  \frametitle{Ergodicity}
  
\end{frame}


\begin{frame}
  \frametitle{Moments}
  
\end{frame}


\begin{frame}
  \frametitle{Estimator}
  
\end{frame}


\begin{frame}
  \frametitle{Convergence}
  
\end{frame}

\begin{frame}
  \frametitle{Biais}
  
\end{frame}


\begin{frame}
  \frametitle{Efficiency}
  
\end{frame}




%% ---------------------------------------------------------------------------
%% End document
%% ---------------------------------------------------------------------------
\end{document}


%% ---------------------------------------------------------------------------
%% Sample code
%% ---------------------------------------------------------------------------

% \begin{frame}
% \frametitle{Example with a Figure}
%     \makebox[\linewidth]{\includegraphics[width=\paperwidth]{../output/step_007_top_ae_dotplot.pdf}}
% \end{frame}


% \begin{frame}
% \begin{adjustwidth}{-2em}{-2em} % Wider frame 
%   \frametitle{Example with a Wide Table}  
%   \setlength\tabcolsep{4pt}  % default value: 6pt
%   %\footnotesize
%   \centering
%   \input{../output/mytable.tex}\\
% \end{adjustwidth}
% \end{frame}








